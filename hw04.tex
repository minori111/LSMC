\documentclass[11pt,a4paper]{article}
\usepackage{amsmath}
\usepackage{amsthm}
\usepackage{amssymb}
\usepackage[margin=2cm]{geometry}
%\usepackage{thmbox}
\usepackage{graphicx}
\usepackage[dvipsnames,usenames]{color}
\usepackage{url}
\usepackage{comment}
\usepackage{amsmath, amsthm, amssymb,enumerate}

%\usepackage{enumerate}
%\usepackage{titlesec}
%\usepackage{Rvector}
%\usepackage{mathabx}
\newcommand{\qrq}{\quad\Rightarrow\quad}
\newcommand{\qarq}{\quad&\Rightarrow\quad}
\newcommand{\alp}{\alpha}
\newcommand{\claim}{{\underline{\it Claim:}}~~}
\newcommand{\dbR}{\mathbb{R}}
\newcommand{\ndimr}{\mathbb{R}^n}
\newcommand{\vare}{\varepsilon}
\newcommand{\since}{\because\;}
\newcommand{\hence}{\therefore\;}
\newcommand{\en}{\par\noindent}
\newcommand{\fn}{\footnotesize}

\newcommand{\sect}[2]{#1~~{\mdseries\tiny(#2)}}

\renewcommand{\(}{\left(}
\renewcommand{\)}{\right)}

\let \ds=\displaystyle

\usepackage{xeCJK}
\setCJKmainfont[AutoFakeBold=5,AutoFakeSlant=.4]{標楷體}

%\usepackage{fancyhdr}
%\pagestyle{fancy}
%\renewcommand{\headrulewidth}{0pt}

\renewcommand{\thesection}{Lecture \arabic{section}}
\renewcommand{\thesubsection}{\Roman{subsection}}

\usepackage[T1]{fontenc}

%%%% F U N C T I O N %%%%%
\newcommand{\abs}[1]{\left|#1\right|}
\newcommand{\norm}[1]{\left\|#1\right\|}
\newcommand{\inn}[1]{\left<#1\right>}
\newcommand{\f}[1]{f\!\left(#1\right)}
\newcommand{\g}[1]{g\!\left(#1\right)}
\newcommand{\h}[1]{h\!\left(#1\right)}
\newcommand{\x}[1]{x\!\left(#1\right)}
\newcommand{\D}[1]{D\!\left(#1\right)}
\newcommand{\N}[1]{N\!\left(#1\right)}
\renewcommand{\P}[1]{P\!\left(#1\right)}
\newcommand{\R}[1]{R\!\left(#1\right)}
\newcommand{\V}[1]{V\!\left(#1\right)}
\newcommand{\function}[2]{#1\!\left(#2\right)}
\newcommand{\functions}[2]{\left(#1\right)\!\left(#2\right)}

\definecolor{light-gray}{gray}{0.95}
\newcommand{\textfil}[1]{\colorbox{light-gray}{\large\color{Red} #1}}


\renewcommand{\title}{Large Sparse Matrix Computations: Homework 05}
\renewcommand{\author}{104021615 黃翊軒\\105021508	陳俊嘉\\105021610 曾國恩}
\renewcommand{\maketitle}{\begin{center}\textbf{\Large\title}\\[6pt] {\author}\\[6pt] {\color{Gray}\footnotesize May 31, 2017}\end{center}}
\newcommand{\blue}[1]{{\color{blue}#1}}


\renewcommand{\labelenumi}{(\alph{enumi})}

\newcommand{\Exercise}[2]{\textbf{Exercise #1.} \textit{#2}}
\newtheorem{exercise}{Exercise}

%\parskip=11pt

\begin{document}

  \maketitle
  
  \setcounter{exercise}{0}


  \begin{exercise}
  \end{exercise}  
  \begin{proof}
  	1. Use $\left\{
  	\begin{array}
  	[c]{c}%
  	F(x)=\frac{1}{2}x^{T}Ax-bx\\
  	F(x_{k})+\frac{1}{2}b^{T}A^{-1}b\leq(\frac{\lambda_{1}-\lambda_{n}}%
  	{\lambda_{1}-\lambda_{n}})^{2}[F(x_{k-1})+\frac{1}{2}b^{T}A^{-1}b]
  	\end{array}
  	\right.  ,$%
  	\begin{align*}
  	F(x_{k})+\frac{1}{2}b^{T}A^{-1}b  & =\frac{1}{2}x_{k}^{T}Ax_{k}-bx_{k}%
  	+\frac{1}{2}b^{T}A^{-1}b\\
  	& =\frac{1}{2}(x_{k-1}^{T}+tr_{k-1}^{T})A(x_{k-1}+tr_{k-1})-b(x_{k-1}%
  	+tr_{k-1})+\frac{1}{2}b^{T}A^{-1}b\\
  	& =\frac{1}{2}(x_{k-1}^{T}Ax_{k-1}+tx_{k-1}^{T}Ar_{k-1}+tr_{k-1}^{T}%
  	Ax_{k-1}+t^{2}r_{k-1}^{T}Ar_{k-1})-(bx_{k-1}+btr_{k-1})\\
  	&\ \ \ 
  	+\frac{1}{2}b^{T}%
  	A^{-1}b\\
  	& =F(x_{k-1})+\frac{1}{2}b^{T}A^{-1}b+\frac{1}{2}(tx_{k-1}^{T}Ar_{k-1}%
  	+tr_{k-1}^{T}Ax_{k-1}+t^{2}r_{k-1}^{T}Ar_{k-1})-btr_{k-1}\\
  	& =F(x_{k-1})+\frac{1}{2}b^{T}A^{-1}b+\frac{1}{2}(tx_{k-1}^{T}A(b-Ax_{k-1}%
  	)+t(x_{k-1}^{T}A^{T}-b^{T})Ax_{k-1}\\
  	&\ \ \ 
  	+t^{2}(x_{k-1}^{T}A^{T}-b^{T}%
  	)A(b-Ax_{k-1}))-bt(b-Ax_{k-1})\\
  	& =F(x_{k-1})+\frac{1}{2}b^{T}A^{-1}b+\frac{1}{2}(t^{2}(x_{k-1}^{T}A^{T}%
  	-b^{T})A(b-Ax_{k-1}))-bt(b-Ax_{k-1})\\
  	& \leq(\frac{\lambda_{1}-\lambda_{n}}{\lambda_{1}-\lambda_{n}})^{2}%
  	[F(x_{k-1})+\frac{1}{2}b^{T}A^{-1}b]
  	\end{align*}%
  	\[
  	\implies\frac{1}{2}((t^{2}(x_{k-1}^{T}A^{T}-b^{T})A-2bt)(b-Ax_{k-1}%
  	))\leq\lbrack(\frac{\lambda_{1}-\lambda_{n}}{\lambda_{1}-\lambda_{n}}%
  	)^{2}-1][F(x_{k-1})+\frac{1}{2}b^{T}A^{-1}b]
  	\]
  \end{proof}


\begin{exercise}
\end{exercise}  
\begin{proof}
	
\end{proof}


\begin{exercise}
\end{exercise}  
\begin{proof}
	
\end{proof}


\begin{exercise}
\end{exercise}  
\begin{proof}
	
\end{proof}



\begin{exercise}
\end{exercise}  
\begin{proof}
	The Chebyshev differential equation is written as%
	\[
	\left( 1-x^{2}\right) \frac{d^{2}y}{dx^{2}}-x\frac{dy}{dx}+n^{2}y=0,\text{ }%
	n=0,1,2,\cdots 
	\]
	
	If we let $x=\cosh t$ we obtain%
	\[
	\frac{d^{2}y}{dt^{2}}-n^{2}y=0
	\]
	
	whose general solution is 
	\[
	y=A\cosh nt+B\cosh nt
	\]
	
	or as 
	\[
	y=A\cosh \left( n\cosh ^{-1}x\right) +B\cosh \left( n\cosh ^{-1}x\right) ,%
	\text{ }\left\vert x\right\vert >1
	\]
	
	or equivalently%
	\[
	y=AT_{n}(x)+BU_{n}(x),\text{ }\left\vert x\right\vert >1
	\]
	
	the function $T_{n}(x)$ is a polynomial. For $\left\vert x\right\vert <1$ we
	have%
	\begin{eqnarray*}
		T_{n}(x)+iU(x) &=&\left( \cos t+i\sin t\right) ^{n}=\left( x+i\sqrt{1-x^{2}}%
		\right) ^{n} \\
		T_{n}(x)-iU(x) &=&\left( \cos t-i\sin t\right) ^{n}=\left( x+i\sqrt{1-x^{2}}%
		\right) ^{n}
	\end{eqnarray*}
	
	from which we obtain%
	\[
	2T_{n}(x)=\left( x+i\sqrt{1-x^{2}}\right) ^{n}+\left( x+i\sqrt{1-x^{2}}%
	\right) ^{n}
	\]
	
	so%
	\[
	T_{n}(x)=\frac{1}{2}\left[ \left( x+i\sqrt{1-x^{2}}\right) ^{n}+\left( x+i%
	\sqrt{1-x^{2}}\right) ^{n}\right] 
	\]
\end{proof}







\end{document} 